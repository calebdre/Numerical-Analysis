\documentclass[11pt]{article}	% Everything after % in a line is comment

% Some commonly used packages. You can add more packages if you need
\usepackage[margin=1in]{geometry}
\usepackage{amsmath,amssymb,amsthm}
\usepackage{graphicx}
\usepackage{url}

\title{Team Project Outline in LaTeX Template}
\author{First1 Last1 \and First2 Last2 \and First3 Last3} 
\date{} % Fill in actual date, or comment this line to show current date.

\begin{document}
\maketitle

\begin{abstract}
It is of common practice to write an abstract that quickly tells readers what
this paper is about and what major findings you have. 
%An abstract is a short introduction to the subject at hand.
An abstract should be one paragraph in length.
Do not go off topic. An abstract should attract someone to read your
paper. Avoid technical jargon and an abundance of symbols.
\end{abstract}

\section{Introduction}
A section briefly introduces the background and purpose of this paper.

\section{Methods in this study}
Show the methods that you're studying in this term paper.
For example, you can write some subsections as follows to 
explain the difference scheme of each method, its complexity, local truncation error, etc.

\subsection{Runge-Kutta method}
This is a subsection about RK2 and RK4 methods.

\subsection{Predictor-Corrector method}
This is a subsection about RK2 and RK4 methods.

We consider the Adams 4th-order Predictor-Corrector method
which uses the Adams-Bashforth 4-step explicit method for prediction and Adams-Moulton 3-step implicit
method for correction. With initial value $w_0=\alpha$, suppose we first generate $w_1,w_2,w_3$
using RK4 method. Then for $i=3,4,\dots,N-1$:
\begin{itemize}
\item Use Adams-Bashforth 4-step explicit method to get a predictor $w_{i+1,p}$:
\begin{equation}\label{eq:p-step}
w_{i+1,p} = w_i + \frac{h}{24} [ 55 f(t_i, w_i) - 59 f(t_{i-1}, w_{i-1}) + 37 f(t_{i-2}, w_{i-2}) - 9 f(t_{i-3},w_{i-3}) ]
\end{equation}
\item Use Adams-Moulton 3-step implicit method to get a corrector $w_{i+1}$:
\begin{equation}\label{eq:c-step}
w_{i+1} = w_i + \frac{h}{24} [ 9 f(t_{i+1},w_{i+1,p}) + 19 f(t_i, w_i) - 5 f(t_{i-1}, w_{i-1}) + f(t_{i-2}, w_{i-2}) ]
\end{equation}
\end{itemize}
The output of iteration $i$ is $w_{i+1}$ after prediction step \eqref{eq:p-step} and correction step \eqref{eq:c-step}.

\section{Numerical experiments}
You will need to demonstrate the performance
of the methods on several (ideally 3 to 5) example IVPs (of your own choice).
You can choose some problems from textbook, but make sure that you 
explicitly state what the problem you chose for each test.

To show the performance, it is often better to use figures
rather than tables (unless there are very few numbers to show). 
For example, you can show the result of RK4
using Figure \ref{fig:rk4_example}. If you have multiple results, you can plot each
with a curve (in different color/line-style/marker type) in the plot. If they are too close, you can consider to 
plot $|w_i-y_i|$, the error of estimate $w_i$ to true solution $y_i=y(t_i)$,
instead of actual $y_i$ and $w_i$. This way, you can see which methods have lower errors (higher accuracy).

\begin{figure}
\centering
\includegraphics[width=.5\textwidth]{rk4_example}
\caption{This is an example plot of RK4 on a test problem.}
\label{fig:rk4_example}
\end{figure}

\section{Discussion}
This is a major part for this project. It should constitute your findings and thoughts. 
Based on the tests you have,
you want to comment on the performance of these methods and how would you
suggest to use in practice. Have extensive discussions with your team members
and give detailed reasonings for your claims. You can cite books, papers, or other
resources, such as \cite{Burden:2011a}. ``References'' part below should show
(only) those you cited in the main paper.


\section{Summary}
A quick summary to conclude the term paper using a paragraph or two.

\bibliographystyle{abbrv}
\bibliography{my_references}

\end{document}

