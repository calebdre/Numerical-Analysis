\documentclass[11pt]{article}	% Everything after % in a line is comment

% Some commonly used packages. You can add more packages if you need
\usepackage[margin=1in]{geometry}
\usepackage{amsmath,amssymb,amsthm}
\usepackage{graphicx}
\usepackage{url}

\title{An Analysis on Direct and Iterative Techniques to Solve Linear Systems}
\author{Ratislav Krylov \and Caleb Lewis}
% \date{} % Fill in actual date, or comment this line to show current date.

\begin{document}
\maketitle

\begin{abstract}
    The need to solve the problem $Ax = b$ comes up in many real-world scenarios.
    As a result, there have been many solutions of varying efficiencies that people
    have employed to solve it. In this paper, we look at 7 of these methods and
    compare their complexity and accuracy.
\end{abstract}

\section{Introduction}
A section briefly introduces the background and purpose of this paper.

\section{Methods}

\subsection{Gauss Elimination}

    \subsubsection{No Pivoting}
    stuff about no Pivoting

    \subsubsection{Partial Pivoting}
    stuff abuot partial Pivoting

    \subsubsection{Scaled Partial Pivoting}
    stuff abuot scaled partial Pivoting


\subsection{Jacobi Iterative}
stuff abuot scaled Jacobi Iterative

\subsection{Jacobi Iterative}
stuff abuot scaled Jacobi Iterative

\subsection{Gauss-Seidel}
stuff abuot Gauss-Seidel

\subsection{Successive Over-Relaxation}
stuff abuot Successive Over-Relaxation

\subsection{Iterative Refinement}
stuff abuot Iterative Refinement

\subsection{(Preconditioned) Conjugate Gradient Method}
stuff abuot (Preconditioned) Conjugate Gradient Method

\section{Numerical experiments}
You will need to demonstrate the performance
of the methods on several (ideally 3 to 5) example IVPs (of your own choice).
You can choose some problems from textbook, but make sure that you
explicitly state what the problem you chose for each test.

To show the performance, it is often better to use figures
rather than tables (unless there are very few numbers to show).
For example, you can show the result of RK4
using Figure If you have multiple results, you can plot each
with a curve (in different color/line-style/marker type) in the plot. If they are too close, you can consider to
plot $|w_i-y_i|$, the error of estimate $w_i$ to true solution $y_i=y(t_i)$,
instead of actual $y_i$ and $w_i$. This way, you can see which methods have lower errors (higher accuracy).

\section{Discussion}
This is a major part for this project. It should constitute your findings and thoughts.
Based on the tests you have,
you want to comment on the performance of these methods and how would you
suggest to use in practice. Have extensive discussions with your team members
and give detailed reasonings for your claims. You can cite books, papers, or other


\section{Summary}
A quick summary to conclude the term paper using a paragraph or two.

\end{document}
